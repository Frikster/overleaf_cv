%% start of file `template.tex'.
%% Copyright 2006-2013 Xavier Danaux (xdanaux@gmail.com).
%
% This work may be distributed and/or modified under the
% conditions of the LaTeX Project Public License version 1.3c,
% available at http://www.latex-project.org/lppl/.


\documentclass[11pt,a4paper,sans]{moderncv}        % possible options include font size ('10pt', '11pt' and '12pt'), paper size ('a4paper', 'letterpaper', 'a5paper', 'legalpaper', 'executivepaper' and 'landscape') and font family ('sans' and 'roman')

% modern themes
\moderncvstyle{banking}                            % style options are 'casual' (default), 'classic', 'oldstyle' and 'banking'
\moderncvcolor{black}                                % color options 'blue' (default), 'orange', 'green', 'red', 'purple', 'grey' and 'black'
%\renewcommand{\familydefault}{\sfdefault}         % to set the default font; use '\sfdefault' for the default sans serif font, '\rmdefault' for the default roman one, or any tex font name
%\nopagenumbers{}                                  % uncomment to suppress automatic page numbering for CVs longer than one page

% character encoding
\usepackage[utf8]{inputenc}                       % if you are not using xelatex ou lualatex, replace by the encoding you are using
%\usepackage{CJKutf8}                              % if you need to use CJK to typeset your resume in Chinese, Japanese or Korean

% adjust the page margins
\usepackage[scale=0.85]{geometry}
%\setlength{\hintscolumnwidth}{3cm}                % if you want to change the width of the column with the dates
%\setlength{\makecvtitlenamewidth}{10cm}           % for the 'classic' style, if you want to force the width allocated to your name and avoid line breaks. be careful though, the length is normally calculated to avoid any overlap with your personal info; use this at your own typographical risks...

\usepackage{import}

% Add picture above name
% \makeatletter
% \@ifpackageloaded{moderncvstylebanking}{%
% \let\oldmakecvtitle\makecvtitle
% \renewcommand*{\makecvtitle}{%
%   {\centering\framebox{\includegraphics[width=\@photowidth]{\@photo}}\par\vspace{10pt}}%
%   \oldmakecvtitle%
% }%
% }{%
% }
% \makeatother
%%%

% Add picture right of name
% \patchcmd{\makehead}
%   {\hfil}
%   {\hspace*{0.15\textwidth}}
%   {}
%   {}
% \patchcmd{\makehead}
%   {\setlength{\makeheaddetailswidth}{0.8\textwidth}}
%   {\setlength{\makeheaddetailswidth}{0.67\textwidth}}
%   {}
%   {}
% \patchcmd{\makehead}
%   {\\[2.5em]}
%   {\hfil\raisebox{-.7cm}{\framebox{\includegraphics[width=\@photowidth]{\@photo}}}\\[2.5em]}
%   {}
%   {}
%%%

% personal data
\name{Cornelis Dirk}{Haupt}
% \title{Curriculum Vitae}                               % optional, remove / comment the line if not wanted
% \address{my address, line 1, line 2, line 3, postcode}{}{}

% optional, remove / comment the line if not wanted; the "postcode city" and and "country" arguments can be omitted or provided empty
\phone[mobile]{+1 604 618 0553}                   % optional, remove / comment the line if not wanted
% \phone[fixed]{01234 123456}                    % optional, remove / comment the line if not wanted
%\phone[fax]{+3~(456)~789~012}                      % optional, remove / comment the line if not wanted
\email{dirk.haupt@gmail.com}                               % optional, remove / comment the line if not wanted
\social[linkedin]{dirk.haupt}
\social[github]{frikster}
\homepage{frikster.github.io}  
% \homepage{www.test.com} 
% optional, remove / comment the line if not wanted
%\extrainfo{additional information}                 % optional, remove / comment the line if not wanted
\photo[64pt][0.4pt]{headshot.jpeg}                  % optional, remove / comment the line if not wanted; '64pt' is the height the picture must be resized to, 0.4pt is the thickness of the frame around it (put it to 0pt for no frame) and 'picture' is the name of the picture file
%\quote{Some quote}                                 % optional, remove / comment the line if not wanted

% to show numerical labels in the bibliography (default is to show no labels); only useful if you make citations in your resume
%\makeatletter
%\renewcommand*{\bibliographyitemlabel}{\@biblabel{\arabic{enumiv}}}
%\makeatother
%\renewcommand*{\bibliographyitemlabel}{[\arabic{enumiv}]}% CONSIDER REPLACING THE ABOVE BY THIS

% bibliography with mutiple entries
%\usepackage{multibib}
%\newcites{book,misc}{{Books},{Others}}
%----------------------------------------------------------------------------------
%            content
%----------------------------------------------------------------------------------
\begin{document}
%\begin{CJK*}{UTF8}{gbsn}                          % to typeset your resume in Chinese using CJK
%-----       resume       ---------------------------------------------------------
\makecvtitle

\small{Software Engineer sensitive to Data Science considerations. Interested in deep learning, neural nets and GPU-accelerated parallel processing.}

\section{Expertise}
  \begin{itemize}
   \item{
     \textbf{Fluent}{: Python \& PyQt, R \& R-Shiny, MATLAB}
      }
   \item{
     \textbf{Proficient}{: C/C\#/Java, Scheme, Haskell, Prolog, Linux, Raspberry Pi}
   }
 \end{itemize}

\section{Education}
  \begin{itemize}
      \item{\cventry{2015 - September 2017}{M.Sc in Neuroscience}{University of British Columbia}{Vancouver, CA}{86.2\%}{}}
    \item{\cventry{2010 - 2014}{B.Sc in Cognitive Systems, Computational Intelligence and Design}{University of British Columbia}{Vancouver, CA}{81.4\%}{}}
    \end{itemize}

  
%     \subsection{University of British Columbia}{Vancouver, CA}
%       {M.Sc in Neuroscience;  86.2\%}{2015 - September 2017}
%     \resumeSubheading
%       {University of British Columbia}{Vancouver, CA}
%       {B.Sc in Cognitive Systems, Computational Intelligence and Design; 81.4\%}{2010 - 2014}
%   \end{itemize}

\section{Experience}
\begin{itemize}
    \item{\cventry{2014 - Present}{Various Positions}{Blusson Spinal Cord Centre (ICORD)}{University of British Columbia}{}{}}
     
 \begin{itemize}
      \item{URP-ShinyApp (\href{https://github.com/Frikster/URP-ShinyApp}{https://github.com/Frikster/URP-ShinyApp})}
          {Developed an online R-Shiny application called URP-ShinyApp to make data mining and predictive modeling easier for subject matter experts in spinal cord clinical research.}
          \begin{itemize}
          \item{The application makes use of the R "party" package to implement a decision tree learning algorithm known as unbiased recursive partitioning for multivariate analysis. Application continues to be used by the Vancouver General Hospital's spinal cord clinic to assess a spinal cord injury patient's eligibility for various clinical trial research settings. In particular, Prof. John Steeves, founding director of the International Collaboration on Repair Discoveries (ICORD), continues to use it extensively to produce figures and reach conclusions.}
          \item{Independent Consultant (2017 - Present)}
          {Part of an international team attempting to find standardized criteria for the enrollment of patients to nerve transfer surgery. Provide statistical analysis and implement feature requests to URP-ShinyApp as requested. Key collaborators include Ida Fox, Project lead and Associate Professor of Surgery at Washington University School of Medicine in St. Louis.}
          \end{itemize}
        \item{Programmer (2015)}
          {Part of an international team that established a working protocol for including spinal cord injury patients in clinical trials. Generated quality reports through decision tree learning and statistical analysis using R. Key collaborators include Prof. Armin Curt, Chief Consultant and Director of the Spinal Cord Injury Centre at Der Balgrist Hospital in Zurich, Switzerland and John Steeves.}
        \item{Research Assistant (2014 - 2016)}
          {Generated quality reports through data mining and statistical analysis using R and URP-ShinyApp to be used at conferences. Statistical techniques included power analysis and rasch models.}
      \end{itemize}

    \item{\cventry{July 2013--August 2013}{Various Positions}{Blusson Spinal Cord Centre (ICORD)}{Test1}{test2}{\vspace{3pt}test3}}
    
    
      {TH Murphy Laboratory}{University of British Columbia}
      {Research Assistant}{2015 - 2017}
      
      
      \begin{itemize}
        \item{Mesoscale Brain Explorer (\href{https://github.com/Frikster/Mesoscale-Brain-Explorer}{https://github.com/Frikster/Mesoscale-Brain-Explorer})}
          {Designed and developed a user-friendly standalone cross-platform pyqt4/pyqtgraph (Python GUI frameworks) application for automating and standardizing the analysis pipeline and visualization output required in TH Murphy's laboratory, thereby making it easier for researchers without a background in programming to conduct their analysis. Application continues to be used regularly by three researchers, one who has used it extensively for seven months and continues to use it daily.
          }
          \item{Data Science}
          {Developed Python scripts for automated neurophotonic data collection using Raspberry Pi coupled with the online data analytics tool Plotly and Bokeh for data visualization and lab reports. Provided evidence against a common assumption that the relationship between neurophotonic signal and light intensity is linear.
          }
          \item{Web Engineering}{Implemented and helped optimize the upload functionality of a neurophysiology data analytics website called \href{SpikeSortingTest.com}{SpikeSortingTest} using the Python web framework Django.}
      \end{itemize}

    \item{\cventry{July 2013--August 2013}{Various Positions}{Blusson Spinal Cord Centre (ICORD)}{Test1}{test2}{\vspace{3pt}test3}}
    
    {Teaching Assistant (various positions)}{University of British Columbia}
      {Teaching Assistant}{2014 - 2015}
      
      
      \begin{itemize}
        \item{Developmental Neuroscience}
          {Facilitated discussions with students, either one to one or in a group setting, on questions and problems posed by students (BIOL 458)}
        \item{Computation in Engineering Design}
          {Assisted in designing and testing course material, assisted in introducing students to the fundamentals of programming using C as a teaching language (APSC 160)}
      \end{itemize}


\end{itemize}

%-----------PROJECTS-----------------
\section{Projects}
  \begin{itemize} 
      \item{Circuit Solver Android App (\href{https://github.com/Frikster/CircuitSolverApp}{https://github.com/Frikster/CircuitSolverApp})}
      {As part of a team, developed an Android application (Java) that uses OpenCV to process images of drawn circuits, TensorFlow to recognize components found by OpenCV and Android NDK and NgSpice open source circuit simulator for solving the resultant circuit given particular inputs. Project includes requirements, design and test plan documentation.}
    \item{Towers VR (\href{https://github.com/Frikster/BMEG554-Towers}{https://github.com/Frikster/BMEG554-Towers})}
      {Developed a virtual reality (VR) environment using Unity3D and Leap Motion for a Biomedical Engineering self-directed study course (BMEG554) to explore the feasibility of using VR and Leap Motion in a clinical settings. Project outline: \href{http://icord.org/2016/05/30518/}{http://icord.org/2016/05/30518/} }
  \end{itemize}



  
  
%-----------PUBLICATIONS-----------------
\section{Publications and Presentations}
  \begin{itemize}
  \item{Thesis}
      {Haupt, C. D. (2017). Mesoscale Brain Explorer, a flexible Python-based image analysis and visualization tool (T). University of British Columbia. Retrieved from \href{https://open.library.ubc.ca/cIRcle/collections/24/items/1.0354398}{https://open.library.ubc.ca/cIRcle/collections/24/items/1.0354398}}
    \item{Peer-reviewed}
      {\textbf{Haupt, D.}, Vanni, M. P., Bolanos, F., Mitelut, C., LeDue, J. M., \& Murphy, T. H. (2017). Mesoscale brain explorer, a flexible python-based image analysis and visualization tool. Neurophotonics, 4(3), 031210. \href{https://www.ncbi.nlm.nih.gov/pubmed/28560240}{https://www.ncbi.nlm.nih.gov/pubmed/28560240}}
    \item{Poster}
      {Mesoscale Brain Imaging: Seed-pixel Correlation Explorer. June 2016 Poster presented at the Neurofutures Conference: Circuit Structure and Dynamics, Allen Institute Seattle, Seattle,  WA. \href{https://www.slideshare.net/slideshow/embed\_code/key/cWfqoJOTrRzFBy}{https://www.slideshare.net/slideshow/embed\_code/key/cWfqoJOTrRzFBy}}
    \item{Poster}
      {The Feasibility of Leap Motion Virtual Reality as a Clinical Tool in Spinal Cord Injury Research. May 2016 Poster presented at Praxis conference, Rick Hansen Institute, Vancouver, BC. \href{https://goo.gl/hULZnc}{https://goo.gl/hULZnc}}
          \item{Peer-reviewed}
          {Murphy, T. H., Boyd, J. D., Bolaños, F., Vanni, M. P., Silasi, G., \textbf{Haupt, D.}, \& LeDue, J. M. (2016). High-throughput automated home-cage mesoscopic functional imaging of mouse cortex. Nature communications, 7. \href{https://www.nature.com/articles/ncomms11611}{https://www.nature.com/articles/ncomms11611}}
          \item{Poster}
      {Differences between $<$2 Week and 1 Month Assessments of Neurological and Functional Capacities for the Reliable Recruitment and Stratification of Incomplete Cervical SCI Trial Participants into Homogeneous Study Cohorts. May 2015 poster presented at the 4th Joint ISCOS/ASIA International Conference in Montreal, Canada. \href{https://www.slideshare.net/DirkHaupt/draft-7-montreal-posters-1-49973542}{https://www.slideshare.net/DirkHaupt/draft-7-montreal-posters-1-49973542}}
          \item{Award Poster}
      {Searching for Trial Endpoints and Stratifying Subpopulations of AIS-D Participants for Clinical Studies. May 2015 poster presented at the 4th Joint ISCOS/ASIA International Conference in Montreal, Canada. \textit{Awarded the 4th ISCos ASIA 2015 Poster Award (top 6) for outstanding research (1 of 6 to receive the award)}. \href{https://www.slideshare.net/DirkHaupt/draft-7-montreal-posters-1-49973542}{https://www.slideshare.net/DirkHaupt/draft-7-montreal-posters-1-49973542}}
  \end{itemize}

References available upon request


% \section{Previous Employment}

% \vspace{6pt}

% \begin{itemize}

% \item{\cventry{July 2013--August 2013}{Construction Site Operative}{Eurogold Groundworks and Civil Engineering}{Moston}{}{\vspace{3pt}I was responsible for the administrative duties and the tidiness and general order of the site. I worked in a safety-oriented manner, often working alongside construction plant and machinery. At the end of my work with the company my colleagues praised my work ethic.}}
% \begin{itemize}
% \item[$\bullet$\scshape\bfseries] tg
% \item[$\bullet$\scshape\bfseries] tset
% \begin{itemize}
% \item test sub
% \end{itemize}
% \item[$\bullet$\scshape\bfseries] test
% \item[$\bullet$\scshape\bfseries] test
% \end{itemize}
% \vspace{6pt}

% \item{\cventry{2009--2011}{Waiter, Catering and Banqueting}{Lavender Best Western Park Hall Hotel}{Charnock Richard}{}{\vspace{3pt}I worked for three years as a team leader providing silver service at weddings, stag/hen nights, and business conferences. I was often trusted with other jobs such as setting up conference rooms, moving beds around the hotel, and guiding big groups of customers to their rooms. As a team leader I would delegate tasks to a team of about 5 people, often new staff who needed training, and lead the group to service. During this time I worked in a highly professional manner and was focused to provide excellent customer service, even during high stress functions and events.}}

% \vspace{6pt}

% \item{\cventry{January 2008}{Assistant Electrician}{Ryan Electricals}{Southport}{}{\vspace{3pt}I spent a week working as an assistant electrician as part of my high school work experience. My roles included wiring plugs, tidying, and assisting with household electrical repairs and fittings.}}

% \end{itemize}

% \section{Education}

% \vspace{5pt}

% \subsection{Academic Qualifications}

% \vspace{5pt}

% \begin{itemize}

% \item{\cventry{2011--2015}{Meng (Hons) Electrical and Electronic Engineering }{Lancaster University}{Lancaster}{\textit{Predicted First Class}}{}}

% \item{\cventry{2009--2011}{A levels}{Runshaw College}{Leyland}{\textit{ICT (A) Physics (B) Maths (B)}}{}}  % arguments 3 to 6 can be left empty

% \item{\cventry{2002--2009}{11 GCSEs}{Southlands High School}{Chorley}{\textit{A* to B Including Maths and English}}{}}

% \end{itemize}

% \vspace{2pt}

% \subsection{Notable Projects}

% \vspace{5pt}

% \begin{itemize}

% \item{\textbf{Masters Project (Ongoing):} \textit{'Development of an Intelligent Humanoid Robot'}

% \vspace{3pt}

% \small{I am part of a team developing a 5ft autonomous humanoid robot. This ambitious project requires strong team-working skills and high technical ability. I work well as part of the team, contributing in group discussions and taking initiative to set myself tasks when the next stage of the project is not clear. Given the role of electronics supervisor I am responsible for setting goals and ensuring all the electronic system designs are realised on time and meet the specifications of the project.}}

% %\newpage

% \item{\textbf{3rd year individual project:} \textit{'Artificial Neural Network Approach to Source Localisation in Radiation Portal Monitoring'}

% \vspace{3pt}

% \small{This challenging project took place over the entirety of my third year. It required excellent planning and organisational skills, and the ability to teach myself an entirely new and complex subject. The project was a success, with the system being able to localise a radioactive source down to $ 3 cm$ within a $ 600 m^3$ sensor array. This project has been suggested for publication by my supervisor.}}

% \vspace{6pt}

% \item{\textbf{Industrial Project with Leyland Motors Ltd:}\textit{'Development of a Facility to Ensure the Achievement of Torque Parameters for a Specific Axle Configuration'}

% \vspace{3pt}

% \small{In the 3rd year of my course I spent a week completing an industrial project for Leyland Motors. I worked with a team operating as consultants for a particular problem the company was having. During this project I was working in a professional environment, and co-operating with various managers and engineers to create a design that met the requirements of the problem.}}

% \end{itemize}

% \section{Technical and Personal skills}

% \vspace{6pt}

% \begin{itemize}

% \item \textbf{Programming Languages:} Proficient in: C, C++, Python, Matlab, Arduino, TeX \\ Also basic ability with: Assembly, VBA, VHDL.

% \vspace{6pt}

% \item \textbf{Industry Software Skills:} SolidWorks (Advanced), Matlab (Advanced), Ansys (Intermediate),  LTspice (Intermediate), Most MS Office products including MS project and MS access (Advanced).

% \vspace{6pt}

% \item \textbf{General Business Skills:} Good presentation skills, Works well in a team.

% \vspace{6pt}

% \item \textbf{Other:} Good soldering and spot welding skills, Can write well organised and structured reports.

% \end{itemize}

% \section{Interests and extra-curricular activity}

% \vspace{6pt}

% \begin{itemize}

% \item{I was a "fresher representative" in my 2nd and 3rd years of university, this required me to guide, look after, and ensure that a particular flat of first years have a good time in their first week, and feel consoled in what for most of them is there first time living away from home. We were responsible for the safety and wellbeing of the group of first years during the first week, and during this time I made good friends with all of them.}

% \vspace{6pt}

% \item{I am a member of a number of university societies. I was also the vice president and co-founder of the flash mob society. My roles in this included recruiting members, in which during "fresher's fair" we enlisted over 200 new members. This was regarded as very successful, considering other societies averaged around 50. I also appeared in an interview on the university television station, set up a society bank account, and helped organise the events. One of these events was featured in the local newspaper.}

% \vspace{6pt}

% \item{I am also an avid hiker, having completed the national 3 peaks challenge last summer. Other interest include guitar, which I am self-taught, and home brewing.}

% \end{itemize}

% \section{References}

% \vspace{6pt}
 
% \begin{itemize}

% \item{Up to 4 references available on request}

% \end{itemize}

% Publications from a BibTeX file without multibib
%  for numerical labels: \renewcommand{\bibliographyitemlabel}{\@biblabel{\arabic{enumiv}}}% CONSIDER MERGING WITH PREAMBLE PART
%  to redefine the heading string ("Publications"): \renewcommand{\refname}{Articles}
\nocite{*}
\bibliographystyle{plain}
\bibliography{publications}                        % 'publications' is the name of a BibTeX file

% Publications from a BibTeX file using the multibib package
%\section{Publications}
%\nocitebook{book1,book2}
%\bibliographystylebook{plain}
%\bibliographybook{publications}                   % 'publications' is the name of a BibTeX file
%\nocitemisc{misc1,misc2,misc3}
%\bibliographystylemisc{plain}
%\bibliographymisc{publications}                   % 'publications' is the name of a BibTeX file

%-----       letter       ---------------------------------------------------------

\end{document}


%% end of file `template.tex'.
